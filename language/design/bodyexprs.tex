\section{Sequential Expressions aka Blocks}

Since everything is an expression in Maxwell, there are no statements. Blocks \lstinline|{ ... }| in classic C-like languages are statements, i.e. they describe a computational step and do not have a value. In Maxwell, blocks also have a return value, depending on the syntax used for their definition. The following possibilities exist:

\emph{Implicit return value}. In its simplest form, a block in Maxwell looks the same as in C. A list of semicolon-separated expressions enclosed in curly braces. The return value corresponds to the last statement in the block.
\lstinline|{ expr0; expr1; ... exprn; } = exprn;|

\emph{Explicit single return value}. The block expression may be prefixed by a return variable name which may be assigned a value inside the block body. The variable may be used like any other variable or function argument, and will be returned as the block's value.
\lstinline|r { ... r = exprn; ... } = r;|

\emph{Explicit tuple return value}. Instead of a single return variable, the block expression may define a tuple of variables that will be returned as value.
\lstinline|(r,s) { ... r = exprn; s = exprm; ... } = (r,s);|

Refer to listing \ref{lst:seqexp_example} for an example of each type of sequential expression.
\begin{lstlisting}[caption=Sequential Expression examples,label=lst:seqexp_example]
// implicit return value
// a = 2, Real
var a = { var k = 4; sqrt(k); };

// explicit single return value
// b = 2, Real
var b = r { r = sqrt(4); };
var b = r Real { r = sqrt(4); };

// explicit tuple return value
// c = (2,4), (Real,Real)
var c = (r,s) { r = 2; s = r+2; };
var c = (r Real, s Real) { r = 2; s = r+2; };
\end{lstlisting}


\subsection{Future Work}
It might actually be possible to unify the definition of variables, function arguments and block return variables into one entity.
