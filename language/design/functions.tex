\section{Functions}

\subsection{Syntax}

Functions represent the core concept of the Maxwell programming language. They map a set of input arguments to an output expression. This expression can be anything from a simple arithmetic operation to a complex algorithm. Unlike most members of the C family of languages, functions in Maxwell are anonymous and need to be assigned to a variable or given a name in order to be called. The syntax for a function definition looks as shown in listing \autoref{lst:functions:def}.

\begin{lstlisting}[label=lst:functions:def, caption=Function definition syntax]
-> expr // no arguments
a0 -> expr // one argument
(a1,a2) -> expr // multiple arguments
\end{lstlisting}

Functions are objects that may either be defined in global scope (as you would do in classic C), or that may be assigned to a variable which may be called later. The two possibilities are outlined in listing \autoref{lst:functions:names}.

\begin{lstlisting}[label=lst:functions:names, caption=Giving names to functions]
square: x -> x*x;
var f = x -> x*x;
// callable as square(...) and f(...)
\end{lstlisting}

Note that the concept of return types the reader might be familiar with from the C language, arises from Maxwell's sequential expression. By using a sequential expression as the function's expression, any number of return types may be realized. Refer to the example in listing \autoref{lst:functions:return}.

\begin{lstlisting}[label=lst:functions:return, caption=Function return values]
square: x -> x*x
fibonacci: (x,y) -> x+y
invsquare: x -> r { r = x*x; r = 1/r; }
swapinc: (x,y) -> (s,t) { s = y+1; t = x+1; }
\end{lstlisting}


\subsection{Bound Arguments}

Maxwell shall support the notion of Bound Arguments. That is, certain arguments of a function may be set to a constant value, thus yielding a new function which lacks those arguments. This concept is similar to Haskell's \emph{currying}. This feature gives rise to contextual function calls (which C++ calls \emph{member function calls}) and encourages software design that relies on callback functions, as well as functional design in general. The concept is illustrated in an example in listing \autoref{lst:functions:binding}.

\begin{lstlisting}[label=lst:functions:binding, caption=Example of function argument binding]
add: (a Int, b Int) -> a+b;
var increment = add.bind("b", 1);
// add is a mapping (Int, Int) -> (Int)
// increment is now a mapping (Int) -> (Int)

var x = add(1,2); // x = 3
var y = increment(1); // y = 2
\end{lstlisting}


\subsection{Contextual Calls}

Consider C++'s member functions. They are functions which have an implicit function argument representing the object they are called on. For example \lstinline{A a; B b; a.add(b);} is restructured to \lstinline{A::add(&a,b)}, implicitly passing \lstinline{a} as the first function argument. A similar concept is possible in Maxwell, yet in a more generic fashion. Typing \lstinline{var x = a.add(b);} in Maxwell encompasses an argument bind as well as a function call. The expression \lstinline{a.add} is restructured to \lstinline{add.bind(0,a)}, thus creating a new function which has its first argument bound to \lstinline{a}. Of course this expression by itself may already be assigned to a variable or given a name in order to call it later. Yet in this example the call expression \lstinline{...(b)} immediately calls the function and returns its result. See listing \autoref{lst:functions:ctxcall} for an example.

\begin{lstlisting}[label=lst:functions:ctxcall, caption=Contextual call examples]
// The following are equivalent:
a.add(b);
add(a,b);
add.bind(0,a)(b);

// Callback functions are easily defined:
var cb = a.add;
cb(5); // actually calls add(a,5)
\end{lstlisting}