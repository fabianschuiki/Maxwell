% Copyright (c) 2014 Fabian Schuiki
\section{Literals}

\begin{verbatim}
number_literal : /[0-9](\.?[0-9\u'])*/
string_literal : /".*(?!\\)"/

array_literal  : "[" expr ("," expr)* "]"
set_literal    : "{" expr ("," expr)* "}"
map_literal    : "{" primary_expr ":" expr ("," primary_expr ":" expr)* "}"
\end{verbatim}

Literals represent values that may be statically derived at compile time.

\begin{description}
	\item[Number literals] start with a digit and may contain digits, periods,
		apostrophes and any letter unicode character. The number format is not
		part of the syntax, since it may be extended by the user.
	\item[String literals] start and end with double quotes, and may contain
		every unicode character. A backslash is used to escape double quotes.
	\item[Array literals] start and end with brackets, and contain one or more
		expressions.
	\item[Set literals] start and end with braces, and contain one or more
		expressions.
	\item[Map literals] start and end with braces, and contain one or more
		key-value-pairs, where the key is a primary expression and the value is
		any expression.
\end{description}

There is no empty array, set or map literal since the empty set and map would be
ambiguous and would collide with the empty block expression.

\subsection{Examples}

\begin{verbatim}
1234
123'456.984
"hello"
"He said \"Hello\"!"
[1,2,3,"four",true,nil]
{1,2,3,"four",true,nil}
{1: "one", "two": 2, nil: 492}
\end{verbatim}
