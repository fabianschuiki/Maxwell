% Copyright (c) 2014 Fabian Schuiki
\section{Definition}

\begin{verbatim}
definition : name ":" expr
\end{verbatim}

% \begin{description}
% \item[expr] can be any valid expression and is evaluated at compile time until a
%             function, a type, or a constant value is obtained. If this is not
%             possible, the compiler will stop with an error.
% \end{description}

A \emph{definition} is an expression with a label assigned to it. The label
allows other parts of the code to refer to the definition. Only definitions are
exported out of packages and can be referenced from other code. Multiple
definitions with the same label are allowed. When other code refers to that
label it is the compiler's responsibility to choose the definition that matches
the required type.

\subsection{Examples}

\begin{verbatim}
(a) A: func x -> x*x
(b) B: func x,y -> x*y
(c) C: type Int range 0 to 7
(d) D: 123
(e) E: A(2)
\end{verbatim}

Where \begin{exdesc}
	\item is a function with one argument,
	\item is a function with two arguments,
	\item is a type representing integers from 0 to 7,
	\item is the integer constant 123, and
	\item is the integer constant 4
\end{exdesc}.
