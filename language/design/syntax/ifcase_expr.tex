% Copyright (c) 2014 Fabian Schuiki
\section{If-Case Expression}

\begin{verbatim}
ifcase_expr : "{" ifcase_cond ("," ifcase_cond)* ("," ifcase_othw)? "}"
ifcase_cond : expr "if" expr
ifcase_othw : expr "otherwise"
\end{verbatim}

An if-case expression is a variant of a regular if expression. It resembles the
standard mathematical notation for distinguishing multiple cases for a variable
or condition. The conditions are checked in the order that they appear, and the
corresponding expression is evaluated and returned for the first condition that
holds. If none hold, the expression of the \emph{otherwise} clause is returned,
or \emph{nil} if none such exists.

\subsection{Examples}

\begin{verbatim}
(a) { "none" if x == 0,
      "some" otherwise }
(b) { 0 if x == 0 }
\end{verbatim}

Where \begin{exdesc}
\item is of type \emph{String}, and returns \verb|"none"| if x is 0,
      \verb|"some"| otherwise; and
\item is of type \emph{String|nil}, and returns 0 if x is 0, \emph{nil}
      otherwise.
\end{exdesc}
