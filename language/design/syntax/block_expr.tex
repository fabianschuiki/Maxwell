% Copyright (c) 2014 Fabian Schuiki
\section{Block Expression}

\begin{lstlisting}[language=EBNF]
block_expr : block_ret? "{" (expr ";")* expr ";"? "}"
block_ret  : ident
           | "(" block_arg ("," block_arg)* ")"
block_arg  : ident (":" type_expr)?
\end{lstlisting}

A \emph{block expression} is a sequence of expressions that are executed in the
order that they appear. The expressions are separated by semicolons, the last of
which is optional. The block may define one or more return variables that are
joined into a tuple and returned as the resulting value of the block. They
behave like every other variable inside the block, allowing expressions to
assign values to them.

\subsection{Behaviour}

An \emph{empty} block results in \lstinline{nil}. A block \emph{without return
variables} returns its very last expression. A block \emph{with one return
variable} returns that variable. A block \emph{with multiple return variables}
returns a tuple of these variables. Note that in the presence of return
variables, the last expression in the block has no significance.

If the block is a function body and contains one or more \emph{return
expressions}, its type is expanded to also include types of all return values,
forming a union type if necessary.

If the block is a loop body and contains one or more \emph{yield expressions},
its type is expanded to also include the yield type, forming a union type if
necessary. The yield type is the array of the union over all yield values.

\subsection{Examples}

\begin{lstlisting}
// implicit return value
(A) {}
(B) { 123 }
(C) { var x = 4; x*2; }

// explicit return value
(D) x { x = 123 }
(E) (x,y) { x = 12; y = 34; }
(F) (s: String) { s = "abc" }

// return and yield
(G) { return "abc"; return nil; 123 }
(H) { yield "abc"; yield nil; 123 }
(I) { yield "abc"; return nil; 123 }
(J) (x,y) { x = 12; y = 34; return "abc"; yield nil; 123 }
\end{lstlisting}

Where \begin{exdesc}
    \item returns nil;
    \item returns the integer 123;
    \item returns the integer 8;
    %
    \item returns an integer;
    \item returns a tuple of integers;
    \item explicitly states its return type;
    %
    \item is of type \lstinline{String|nil|Int};
    \item is of type \lstinline{Array[String|nil]|Int};
    \item is of type \lstinline{Array[String]|nil|Int}; and
    \item is of type \lstinline{(Int,Int)|String|Array[nil]|Int}.
\end{exdesc}

\begin{lstlisting}
// concrete example
r := for (f in files) {
    if (f.exists?)
        yield f.size;
    else
        return Error("file %{f} does not exist");
}
\end{lstlisting}

In this case the block is the body of the for loop. It contains a yield
expression with an integer value and a return expression with an Error value.
Thus the variable r is of type \lstinline{Array[Int]|Error}. If all files exist
the yield expression appends each file's size to the result of the for loop,
assembling an array of integers in the process. However if a file does not
exist, the block returns an error which discards all previous yields and aborts
the loop.
