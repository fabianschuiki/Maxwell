% Copyright (c) 2014 Fabian Schuiki
\section{Function Expression}

\begin{verbatim}
func_expr    : "func" func_variant ("," func_variant)*
func_variant : pattern_expr "->" expr
\end{verbatim}

A \emph{function expression} is a mapping from a pattern to a an expression. In
terms of other languages, the pattern may be thought of as the arguments and the
expression as the return value. A function may have multiple variants, each
of which maps different pattern to an expression --- much like function
overloading in C++. The value of the expression is the resulting function object
which is of function %type (\lstinline{A -> B}).

\subsection{Examples}

\begin{verbatim}
(a) func x -> x*x
(b) func x,y -> x*y
(c) func nil -> 1234
(d) func (x: Int, y: Int) -> x+y
(e) func x -> { x*x; }
(f) func x -> u,v { u = x; v = x*x; }
(g) func Point(x,y) -> x*x+y*y
(h) func 0 -> "zero",
         x -> "anything"
\end{verbatim}

Where \begin{exdesc}
	\item is a function that maps the value x to its square;
	\item maps the tuple (u,v) to the result of multiplying u and v;
	\item maps to the constant integer 1234;
	\item maps the tuple (x,y) to the sum of x and y, where x and y both are
	      integers;
	\item maps the value x to a block expression that results in the square of
	      x;
	\item maps the value x to the tuple (u,v), where u is the same as x and v is
	      the square of x;
	\item maps a \texttt{Point}, whose coordinates are captured as x and y,
	      to the sum of its squared components; and
	\item demonstrates a function with multiple variants.
\end{exdesc}
